%%% Template ini dibuat oleh kamhar ngado menggunakan lisensi GPL Versi 2 
%%%% link download https://github.com/kamhar-ngado/presentasiLatex




\documentclass[compress,red]{beamer}
\usetheme{cambridgeUS}

\usepackage{lipsum}
\usepackage{amsmath, amsfonts, amssymb}
\usepackage{graphicx}
\usepackage{xcolor}
\usepackage{fancybox}


\title[LMS U Maxima 4 Calculus]{The Development of Learning Management System Using Maxima for Calculus Material Oriented to Mathematical Thinking and Problem Solving}
\author[uny.ac.id]{Kamhar Ngado}
\institute[\tiny{Yogyakarta State University}]{Yogyakarta State University}
\date {\today}
\logo{{\tiny{\textit{Taqwa, Mandiri, Cendekia}}}\includegraphics [scale=.1]{uny.jpg}}



\begin{document}
	
	\begin{frame}[t]
		\titlepage
	\end{frame}
	

	\begin{frame}[t] {Daftar Isi}
	\tableofcontents 
	\end{frame}

	\begin{frame}{animation}
		\begin{enumerate}
		\setbeamercovered{transparent}
		\item \uncover<1>{enum1}			
		\item \uncover<2>{enum2}
		\item \uncover<3>{enum3}			
		\item \uncover<4>{enum4}
	\end{enumerate}
	\end{frame}


	\section{kamal handsome jump to introduction}%1
	\begin{frame}[t] {Introduction}
	content...
		\begin{enumerate}[1] % [A] maka muncul A dan jika [1] maka muncul 1
			\item [1.] enum 1
			\item [2.] enum 2
			
			\begin{itemize}
				\item [a.] first
				\item [b.] second
			\end{itemize}
			\item [3.] enum 3
		\end{enumerate}
	\end{frame}

	\section{kamal handsome jump to block}%2
	\begin{frame}[t] {Block} %% t = top 
		\begin{block}{Block title}
			content...
		\end{block}
		\vfill %untuk enter
		\begin{exampleblock} {title of example block}
			text....
		\end{exampleblock}
		\vfill
		\begin{exampleblock} {title of example block}
		text....
		\end{exampleblock}
	\end{frame}

	\section{ipsum}
	\begin{frame}[t] {ipsum}
	\lipsum [1]

	\end{frame}
	
	\begin{frame}[t]{Memasukan Gambar}
\begin{center}
	Gambar 1. Logo UNY
\end{center}	
	\begin{figure}
	\centering
	\includegraphics[width=0.3\linewidth]{uny}
	%\caption{}
	\label{fig:uny}
\end{figure}
\end{frame}

	\section{Dummy text \LaTeX}
	\begin{frame}[t] {Dummy text \LaTeX}
		  \textbf{Berpikir matematis}, menurut Mason, Burton, dan Stacey (1982), adalah \lipsum[2]
	\end{frame}

	\section{Thank You}
	\begin{frame}[t] {The end}
	\begin{center}
	\LARGE{Thank You}\\email:\underline{kamhar.ngado.id@gmail.com}
	\end{center}
	\end{frame}


\end{document}